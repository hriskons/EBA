\chapter{Μαθηματικό Υπόβαθρο}
\label{ch:3.chapterMathBackground}

\section{Αραιοί Πίνακες στα Γραμμικά Συστήματα}

Με το όρο Αραιοί Πίνακες \textlatin{(Sparse Matrices)} αναφερόμαστε σε πίνακες όπου τα περισσότερα στοιχεία τους είναι 0. Για την επιστήμη της πληροφορικής, οι πίνακες αυτοί έχουν ιδιαίτερο ενδιαφέρον καθώς μας δίνουν την δυνατότητα να συμπιέσουμε την πληροφορία που περιέχουν σε μικρότερο χώρο από ότι θα χρειάζονταν ένας πυκνός \textlatin{(dense)} πίνακας. Επίσης, μπορούμε να εκμεταλλευτούμε την ιδιότητα που έχουν οι πίνακες αυτοί και να προσεγγίσουμε μια λύση του γραμμικού συστήματος πιο γρήγορα απο ότι θα κάναμε σε έναν πυκνό \textlatin{(dense)} πίνακα.

\subsection{Αποθήκευση και απεικόνιση Αραιών Πινάκων}

Όπως τονίσαμε, η ιδιότητα των αραιών πινάκων να έχουν τα περισσότερα στοιχεία τους 0, μας δίνει την δυνατότητα να εφαρμόσουμε πιο αποτελεσματικούς τρόπους αποθήκευσης απο το να κρατούσαμε όλα τα στοιχεία (μηδενικά και μη μηδενικά) στη μνήμη.

Ύπάρχουν διάφοροι τρόποι αποθήκευσης. Στα πλαίσια της διπλωματικής και λόγω της εξωτερικής βιβλιοθήκης που χρησιμοποιούμε, θα σταθούμε σε 2 βασικές μεθόδους απεικόνισης των αραιών πινάκων.

\subsubsection{Τρίδυμη Δομή Απεικόνισης} \label{TripletForm}
Η δομή αυτή~\cite{davis2006direct} αποτελεί τον πιο απλό τρόπο απεικόνισης των αραιών πινάκων σε ένα υπολογιστικό σύστημα. Για κάθε μη μηδενικο στοιχείο του πίνακα, κρατάμε την γραμμή, τη στήλη και τη τιμή του στοιχείου.

Εκ πρώτης όψεως θα μπορούσε να παρατηρήσει κάποιος οτι χρειαζόμαστε 3 στοιχεία στη μνήμη μας για να απεικονίσουμε ένα μη μηδενικό στοιχείο. Δεν είναι όσο αποδοτικό όσο η απλή απεικόνιση που έχουμε για έναν πυκνό πίνακα όπου κρατάμε όλα τα στοιχεία στη μνήμη.

Το σημαντικό στοιχείο στην απεικόνιση ενός αραιού πίνακα είναι οτι τα στοιχεία που θα χρειαστεί να αποθκεύσουμε θα είναι πολύ λιγότερα απο το μέγιστο αριθμό στοιχείων που μπορεί να έχει ένας πίνακας μεγέθους $n x m$.


Παράδειγμα απεικόνισης αραιού πίνακα σε Τρίδυμη Δομή\\

\selectlanguage{english}
\begin{figure}[ht]
\centering
$\begin{matrix}
0 & 0 & 11.0\\
0 & 3 & 14.0\\
0 & 5 & 16.0\\
1 & 1 & 22.0\\
1 & 3 & 25.0\\
1 & 5 & 26.0\\
2 & 2 & 33.0\\
2 & 3 & 34.0\\
2 & 5 & 36.0\\
3 & 0 & 41.0\\
3 & 2 & 43.0\\
3 & 3 & 44.0\\
3 & 5 & 46.0
\end{matrix}$
\end{figure}
\selectlanguage{greek}

\\
Κάθε γραμμή του παραπάνω πίνακα έχει την παρακάτω μορφή:\\
\textlatin{I J A(I, J)}\\
όπου τα \textlatin{I} και \textlatin{J} είναι οι δείκτες για τον \textlatin{x} και \textlatin{y} άξονα αντίστοιχα και είναι με βάση το 0.

\subsubsection{Δομή της Συμπιεσμένης Στήλης \textlatin{Compressed Column Storage (CCS)}} \label{CCS}
Η δομή αυτή αποτελεί τον πιο συνηθισμένο τρόπο απεικόνισης ενός αραιού πίνακα σε ένα υπολογιστικό σύστημα. Η δομή αυτή αναφέρεται και ως \textlatin{Harwell-Boeing} απεικόνιση~\cite{duff1989sparse}. Σε σχέση με την Τρίδυμη Δομή Απεικόνισης, προσφέρει τη δυνατότητα να αποθηκεύσουμε μόνο τα μη μηδενικά στοιχεία ενός αραιού πίνακα, αλλά χωρίς να χρειαζόμαστε 3 στοιχεία της μνήμης για την απεικόνιση ενός στοιχείου του πίνακα.
Συνολικά για την απεικόνιση ενός αραιού πίνακα Α στη μορφή Συμπιεσμένης Στήλης \textlatin{(CCS)} χρειαζόμαστε 3 διαφορετικούς μονοδιάστατους πίνακες:
\begin{itemize}
  \item Πίνακας Τιμών: περιέχει τις τιμές του πίνακα
  \item Πίνακας Δεικτών των γραμμών: Περιέχει τους δείκτες για όλες τις γραμμές που υπάρχουν μη μηδενικά στοιχεία
  \item Πίνακας Δεικτών Στηλών: Στον πίνακα αυτό βρίσκονται οι δίκτες για τα στοιχεία στον Πίνακα τιμών που ξεκινάει μια στήλη του πίνακα Α. 
\end{itemize}

\subsection{Αλγοριθμική επίλυση αραιών πινάκων}
Η επίλυση αραιών πινάκων μπορεί να πραγματοποιηθεί είτε με την χρήση άμεσων επιλύσεων \textlatin{(direct solvers)} είτε με τη χρήση επαναληπτικών μεθόδων.

Οι μέθοδοι που χρησιμοποιούνται για την άμεση επίλυση γραμμικών συστημάτων που χρησιμοποιούν αραιούς πίνακες βασίζονται στην απαλοιφη κατα \textlatin{Gauss}.

Τα τελευταία χρόνια έχει εδραιωθεί η επίλυση γραμμικών συστημάτων αραιών πινάκων με τη χρήση επαναληπτικών μεθόδων. Οι μέθοδοι αυτοί επιτυγχάνουν την γρηγορότερη και πιο αποδοτική επίλυση των γραμμικών συστημάτων με τη χρήση των παρακάτω μεθοδολογιών:
Μετάθεση των εξισώσεων και των αγνώστων έτσι ώστε να εγγυηθούμε μια σταθερή \textlatin{LU} αποσύνθεση του πίνακα
Οργάνωση των υπολογισμών με βάση τη διαθέσιμη υπολογιστική ισχύ του υλικού \textlatin{(hardware)} στο οποίο πραγματοποιείται η επίλυση. Τα τελευταία χρόνια έχει εδραιωθεί η χρήση πολυνηματικών υπολογιστικών μονάδων απο τη μεριά του επεξεργαστή αλλά και της κάρτας γραφικών (με τη χρήση τεχνολογιών τύπου \textlatin{CUDA, OpenCL}) που επιτρέπουν τη μαζική και παράλληλη επίλυση γραμμικών συστημάτων.

\section{Μέθοδοι Παραγοντοποίησης Πινάκων}

Η βασική μέθοδος επίλυσης ενός συστήματος γραμμικών εξισώσεων $Ax = b$ είναι ο πολλαπλασιασμός του δεξιού μέλους με τον αντίστροφο $x = A^Tb$ (όταν ο $A$ έχει ορθογώνιες στήλες). Ο περιορισμός της ορθογωνικότητας των στηλών καθώς και η μη αποδοτική επίλυση του συστήματος, καθιστά την απλή αυτή λύση μη βιώσιμη για γραμμικά συστήματα που συναντάμε σε πραγματικά προβλήματα.

Ένα βασικό εργαλείο για την επίλυση γραμμικών συστημάτων (ή για την προσέγγιση μιας λύσης) είναι η παραγοντοποίηση ενός πίνακα. Η βασική αρχή της παραγοντοποίησης είναι η αποσύνθεση ενός πίνακα σε γινόμενο απο πίνακες. Ανάλογα με την εκάστοτε μέθοδο παραγοντοποίησης, οι επιμέρους πίνακες έχουν και διαφορετικές ιδιότητες τις οποίες μπορούμε να αξιοποιήσουμε κατα την επίλυση του γραμμικού συστήματος.

Η πιο βασική μέθοδος παραγοντοποίησης είναι η απολοιφή κατα \textlatin{Gauss}. Η απαλοιφή αυτή μπορεί να εφαρμοστεί σε αντιστρέψιμους πίνακες όπου ο $A$ μετατρέπεται σε έναν άνω τριγωνικό U πίνακα ή παραγοντοποιείται σε ένα γινόμενο πινάκων $LU$ όπου $L$ είναι ο κάτω τριγωνικός και $U$ ο άνω τριγωνικός. Η μέθοδος αυτή αν και πολύ διαδεδομένη και χρήσιμη, έχει σαν βασικό μειονέκτημα οτι μπορεί να είναι "ασταθής" για ορισμένα γραμμικά συστήματα. Επίσης, είναι δύσκολο να προσδιορίσουμε το σφάλμα που έχει προκύψει κατα την παραγοντοποίση.

Τα παραπάνω προβλήματα που εμφανίζει η απαλοιφή κατά \textlatin{Gauss} έρχεται να αντιμετωπίσει η Μέθοδος Παραγοντοποίησης $QR$.

Στα πλαίσια των αλγορίθμων που θα αναλύσουμε, γίνεται χρήση της $QR$ παραγοντοποίησης η οποία μπορεί να επιτευχθεί με τη χρήση διάφορων μεθόδων.

Η $QR$ παραγοντοποίηση χρησιμοποιείται κυρίως για την επίλυση γραμμικών συστημάτων, προβλήματα με ιδιοτιμές και προσέγγιση λύσης με βάση την μέθοδο ελαχίστων τετραγώνων.

\subsubsection{Μέθοδος Παραγοντοποίησης \textlatin{QR}}

Ορίζουμε την Παραγοντοποίηση $QR$ ως εξής:\\

\selectlanguage{english}
\theoremstyle{definition}
\begin{definition}{QR}
\selectlanguage{greek}
Για πίνακα $Α$ διαστάσεων $m * n$ όπου $m >= n$ και έχει γραμμικά
ανεξάρτητες στήλες. Τότε υπάρχει ένας μοναδικός $m * n$ πίνακας $Q$ με $Q^TQ=D$,
$D=diag(d_1 ,...,d_n),  d_k >0, k=1,...,n$ και ένας μοναδικός άνω τριγωνικός πίνακας $R$ με
$r_kk =1, k=1,2,...,n$ έτσι ώστε $A=QR$

Η Παραγοντοποίηση $QR$ θα δούμε στην συνέχεια οτι αποτελεί βασικό στοιχείο στον $EBA$ αλγόριθμο καθώς χρησιμοποιείται σε κάθε επαναληπτικό βήμα.


\subsubsection{Επίλυση Παραγοντοποίησης QR με χρήση \textlatin{Householder} μεθόδου} \label{HouseholderMethod}

Η χρήση \textlatin{Householder} μεθόδου μας δίνει τη δυνατότητα να επιλύσουμε το ζητούμενο γραμμικό σύστημα πετυχαίνοντας μίκροτερες αποκλίσεις στα αποτελέσματα σε κάθε βήμα με αποτέλεσμα να αποφύγουμε την εύρεση $Q$ πίνακα ο οποίος δεν είναι ορθογώνιος ~\cite{trefethen1997numerical}.

Η ιδέα πίσω απο την μέθοδο \textlatin{Householder} είναι να σχεδιάσουμε επαναληπτικά πίνακες $Q_1,...,Q_n$ οι οποίοι σταδιακά μετατρέπουν τον $A$ πίνακα σε άνω τριγωνικό πίνακα:

\selectlanguage{english}
A = \begin{bmatrix}
x & x & x\\
x & x & x\\
x & x & x\\
x & x & x\\
\end{bmatrix} \quad \rightarrow \quad $ Q_1A$ = A = \begin{bmatrix}
x & x & x\\
0 & x & x\\
0 & x & x\\
0 & x & x\\
\end{bmatrix}  \quad \rightarrow  \quad $Q_2Q_1A$ = A = \begin{bmatrix}
x & x & x\\
0 & 0 & x\\
0 & 0 & x\\
0 & 0 & x\\
\end{bmatrix}  \quad \rightarrow \quad 

\begin{center}
$Q_3Q_2Q_1A$ = A = \begin{bmatrix}
x & x & x\\
0 & 0 & x\\
0 & 0 & x\\
0 & 0 & 0\\
\end{bmatrix}
\end{center}
\selectlanguage{greek}


όπου $x$ είναι μια μηδενική τιμή. Σημειώνουμε οτι ο $Q_k$ χρησιμοποιεί τις γραμμές $k:m$ και δεν πειράζει καθόλου τις πρώτες $k - 1$ γραμμές και στήλες. 
Συνεπώς η μορφή του πίνακα $Q$ κατα την επανάληψη $k$ θα είναι η εξής:

\[Q_k = \begin{bmatrix}
Ι_{κ-1} & 0 \\
0 & F \\
\end{bmatrix}\]

όπου $I_{k-1}$ είναι ένας $(k-1) * (k-1)$ μοναδιαίαος πίνακας και ο $F$ έχει την εξής ιδιότητα:

\[Fx = \| x \|e_1\]

έτσι ώστε να εισαγάγει μηδενικά στο κάτω μέρος της στήλης $k$.

Ο πίνακας $F$ καλείται και \textlatin{Housholder} ανακλαστήρας (\textlatin{Housholder reflector})

Για $u = \|x\|e_1 - x$ και $P = \frac{uu^*}{u^*u}$ τότε έχουμε:

\[F = I - 2P = I - 2\frac{uu^*}{u^*u}\]

Για το $u$ επιλέγουμε το εξής:

\[u = x + sign(x(1))\|x\|e_1\]

Κατά την ολοκλήρωση της μεθόδου, ο πίνακας $A$ περιέχει τον πίνακα $R$ της Παραγοντοποίησης $QR$ και τα διανύσματα $u_1, ..., u_n$ είναι τα διανύσματα αντανάκλασης τα οποία θα χρησιμοποιηθούν επαναλητπτικά για να σχηματίσουμε το $Q*b$ σε ένα γραμμικό σύστημα. 

\section{Χρήση προρυθμιστών με χρήση επαναληπτικών μεθόδων επίλυσης}

Κατά την επίλυση μεγάλης κλίμακας γραμμικών συστημάτων με τη χρήση επαναληπτικών μεθόδων προτείνεται η χρήση προρυθμιστών έτσι ώστε να επιτευχθεί γρηγορότερη και πιο αποδοτική λύση.

Υπάρχουν επιλύσεις οι οποίες δεν χρησιμοποιούν προρυθμιστές αλλά δεν είναι ανταγωνιστικοί και βιώσιμοι σε προβλήματα μεγάλης κλίμακας που βρίσκουμε σε πραγματικά προβλήματα. Μια τέτοια επίλυση είναι η Επαναληπτική Υπερχαλαρωση\textlatin{(Successive Overralaxation (SOR))}~\cite{young1954iterative} μέθοδος η οποία αποτελεί προσέγγιση της \textlatin{Gauss–Seidel (Liebmann)}. Στη μέθοδο αυτή για γραμμικό σύστημα
\[Ax = b\]
					
Χωρίζουμε τον πίνακα Α σε τρεις(3) υποπίνακες
\[Α = D + L + U\]

όπου 

\begin{itemize}
  \item $A$: περιέχει τις τιμές του πίνακα
  \item $D$ : Πίνακας που περιλαμβάνει μόνο τη διαγώνιο του Α
  \item $L$ : Κάτω τριγωνικός πίνακας που περιλαμβάνει όλα τα στοιχεία απο τη διαγώνιο και κάτω του πίνακα $Α$
  \item $U$ : Άνω τριγωνικός πίνακας που περιλαμβάνει όλα τα στοιχεία απο τη διαγώνιο και πάνω
\end{itemize}

Για την επίλυση του συστήματος χρησιμοποιούμε επαναληπτική μέθοδο που αξιοποιεί τιμές του $x$ απο προηγούμενο βήμα για την εύρεση του $x$ στο τρέχον βήμα.

Οι πιο σύγχρονες μέθοδοι που χρησιμοποιούνται για την επίλυση μεγάλης κλίμακας γραμμικά συστήματα χρησιμοποιούν προρυθμιστές υποχωρου \textlatin{Krylov}. 

\section{Υποχώροι \textlatin{Krylov}} \label{KrylovSubspaces}

Οι υποχώροι \textlatin{Krylov} εμφανίζονται σε πολλές επαναληπτικές μεθόδους για την επίλυση γραμμικών συστημάτων αραιών πινάκων. 

Μπορούμε να ορίσουμε τους υποχώρους \textlatin{Krylov} ώς εξής:


\selectlanguage{english}
\theoremstyle{definition}
\begin{definition}{Krylov}
\selectlanguage{greek}
Για μη-αντιστρέψιμο πίνακα $A \in C^{NxN}$ και  $y \neq 0 \in C^{N}$ o Ν-οστος \textlatin{Krylov} (υπο)χώρος $K_{n}(A,y)$ που παράγεται απο τον $Α$ στον $y$ είναι
\[ Κ_{n} := K_{n}(A,y) := span(y,Ay,...,A^{n-1}y) \]


Χαρακτηριστικά που μπορούμε να παρατηρήσουμε στον παραπάνω ορισμό είναι τα παρακάτω:
\[Κ_1 \subseteq  K_2 \subseteq  K_3 \subseteq ... \]
Οι διαστάσεις αυξάνονται κατά μια σε κάθε βήμα.
\end{definition}
