% \selectlanguage{greek}
% \chapter{Πειραματική Διαδικασία}
% \label{ch:chapterMatrixMarketFormat}

\selectlanguage{greek}
\chapter{Μορφή αρχείων εισόδου \selectlanguage{english}Matrix Market\selectlanguage{greek}}
\label{ch:chapterMatrixMarketFormat}

\section {Περιγραφή μορφής αρχείου}
Η μορφή αρχείων εισόδου \selectlanguage{english}Matrix Market\selectlanguage{greek} ~\cite{boisvert1996matrix} απολεί μια βασική μορφή απεικόνισης πινάκων σε μορφή \selectlanguage{english} ASCII \selectlanguage{greek} αρχείων. Το βασικό χαρακτηριστικό της μορφής αυτής είναι η απλότητα που τη διακρίνει κατα την περιγραφή ενός πίνακα. Ο στόχος της μορφής αυτής είναι να είναι εύκολα προσβάσιμο στη κατανόηση όσο απο ανθρώπους όσο και απο προγράμματα αλλά χωρίς να χάνεται η προσαρμόστηκότητα σε εφαρμογές με πιο αυστηρές δομές ή η επεκτασιμότητα σε συναφείς δομές δεδομένων (όπως είναι η Τρίδυμη Δομή Απεικόνισης \ref{TripletForm} και Δομή της Συμπιεσμένης Στήλης\ref{CCS}).

Στα πλαίσια της παρούσας Διπλωματικής Εργασίας, θα εξετάσουμε δύο απο τις βασικές μορφές δήλωσης πινάκων:

\begin{itemize}
    \item \textbf{Μορφή Συντεταγμένων \selectlanguage{english} Coordinate Format \selectlanguage{greek}} \\
    Είναι η βασική μορφή που χρησιμοποιείται για την αναπαράσταση αραιών πινάκων. Η μορφή αυτή περιλαμβάνει μόνο τις μη-μηδενικές τιμές μαζί με τις συντεταγμένες τους στον πίνακα. Όπως συμπεραίνουμε, η μορφή αυτή είναι ιδανική μόνο όταν τα μη-μηδενικά στοιχεία είναι λιγότερα απο τα μηδενικά.
    \item \textbf{Μορφή Πίνακα \selectlanguage{english} Array Format \selectlanguage{greek}} \\
    Είναι η βασική μορφή που χρησιμοποιείται για την αναπαράσταση πυκνών πινάκων. Στη μορφή αυτή, παρέχουμε όλες τις τιμές (μηδενικές και μη-μηδενικές) σε μια προδιατεταγμένη (ανα στήλη) σειρά.
\end{itemize}

Τα αρχεία εισόδου που διαχειριζόμαστε στην υλοποίηση μας είναι της πρώτης μορφής. Παραθέτουμε ένα απλό παράδειγμα πως ένας αραιός πίνακας αντικατοπτρίζεται σε αυτή τη μορφή:

\begin{center}
\begin{tabular}{ c c c c}
 10 & 15 & 0 & 0 \\ 
 0 & 2.34 & 0 & 221 \\  
 1.32 & 0 & 0 & 23.324 \\
 0 & 0 & 0 & 98.58
\end{tabular}
\end{center}

Το παραπάνω παράδειγμα είναι απο έναν γενικό αραιό πίνακα $4x4$ με πραγματικές τιμές.

Στη μορφή \selectlanguage{english}Matrix Market\selectlanguage{greek} το παραπάνω παράδειγμα θα αναπαριστώνταν ως εξής:
\selectlanguage{english}
\begin{verbatim}
%%MatrixMarket matrix coordinate real general
%===============================================================
4  4  7
1     1   10
1     2   15
2     2   2.34
2     4   221
3     1   1.32
3     4  23.324
4     4   98.58
\end{verbatim}
\selectlanguage{greek}
Η πρώτη γραμμή του αρχείου περιλαμβάει τον τύπο της μορφής του αρχείου. Στην συγκεκριμένη περίπτωση, σημειώνει οτι το αντικείμενο εντός του αρχείου αντιπρωσοπεύει έναν πίνακα σε Μορφή Συντεταγμένων \selectlanguage{english} Coordinate Format \selectlanguage{greek} και οι τιμές που περιέχονται εντός του πίνακα είναι πραγματικές σε κανονική μορφή.

Υπάρχουν διάφορες παραλαγές της παραπάνω μορφής που περιλαμβάνουν μηγαδικές αλλά και ακέραιες τιμές καθώς και μορφές που οι θέσεις απο τις μη-μηδενικές τιμές περιγράφονται βάση προτύπου (\selectlanguage{english}pattern matrices\selectlanguage{greek}).

Επίσης, επιπλέον παράμετροι μπορούν να χρησιμποιηθούν για να δηλώσουμε συμμετρίες στον πίνακα οι οποίες μπορούν να μειώσουν ριζικά το μέγεθος των δεδομένων. Τέτοιες παράμετροι είναι \selectlanguage{english} symmetric, skew-symmetric, Hermitian \selectlanguage{greek}. Στις περιπτώσεις αυτές μόνο οι τιμές για το κάτω τριγωνικό μέρος του πίνακα χρειάζεται να βρίσκονται εντός του αρχείου.

Στη συνέχεια, στο παραπάνω παράδειγμα βλέπουμε οτι η πρώτη γραμμή μετά τα σχόλεια (τα οποία ξεκινάνε πάντα με τον χαρακτήρα \% στην αρχή) έχουμε μια γραμμή η οποία έχει τη μορφή:\\

\begin{center}
\begin{tabular}{ c c c }
 Αριθμός Γραμμών & Αριθμός Στηλών & Αριθμός Μη-Μηδενικών Στοιχείων
\end{tabular}
\end{center}

Η παραπάνω γραμμή μας βοηθάει στο να δεσμεύσουμε με ακρίβεια τον χώρο που θα χρειαστέι το πρόγραμμα μας για να αποθκεύσει τον πίνακα. 

Στη συνέχεια, όλες οι υπόλοιπες γραμμές έχουν την ίδια μορφή η οποία είναι η εξής:

\begin{center}
\begin{tabular}{ c c c }
 Συντεταγμένη Γραμμής & Συντεταγμένη Στήλης & Τιμή
\end{tabular}
\end{center}

Κάθε μια τιμή που λαμβάνουμε απο το αρχείο την τοποθετούμε εντός του πίνακα που έχουμε στην μνήμη.

