\selectlanguage{greek}
\chapter{Βιβλιοθήκη \selectlanguage{english}CSparse\selectlanguage{greek}}
\label{ch:chapterCSparse}

Στο παρόν κεφάλαιο θα αναφέρουμε επιγραμματικά τη χρήση της βιβλιοθήκης \selectlanguage{english}CSparse\selectlanguage{greek} σαν εξωτερική βιβλιοθήκη για την απεικόνιση αραιών πινάκων και πραγματοποίηση πράξεων και μετασχηματισμών πάνω σε αυτούς~\cite{davis2006direct}.

Οι συναρτήσεις της βιβλιοθήκης \selectlanguage{english}CSparse\selectlanguage{greek} χωρίζονται σε 3 βασικές κατηγορίες: τις πρωτογενής, τις δευτερογενής και τις τριτογενής.

Οι πρωτογενής συναρτήσεις περιλαμβάνουν όλες τις συναρτήσεις που χρειάζεται μια εφαρμογή για να δημιουργήσει έναν πίνακα $A$, να λύσει ένα σύστημα $Ax=b$ και γενικά να εφαρμόσει βασικές μεθόδους πινάκων.

Οι δευτερεύουσες συναρτήσεις περιλαμβάνουν ταξινόμιση και παραγοντοποίηση πινάκων, μετασχηματισμοί για πίνακες μεταθέσεων κλπ.
% secondary routines include the ordering and
% factorization methods, forward/backsolve, rank-1 update/downdate, and routines
% for manipulating permutation vectors.

Οι τριτογενής συναρτήσεις έχουν δημιουργηθεί κυρίως για εσωτερική χρήση εντός της βιβλιοθήκης.

Κάθε πρόγραμμα το οποίο χρησιμοποιεί τη \selectlanguage{english}CSparse\selectlanguage{greek} βιβλιοθήκη πρέπει να περιλαμβάνει το \selectlanguage{english}header\selectlanguage{greek} αρχείο: 
\selectlanguage{english}
\begin{verbatim}#include "cs.h"\end{verbatim}

\selectlanguage{greek}
Στο αρχέιο αυτό βρίσκονται οι δηλώσεις όλων των συναρτήσεων που παρέχει η βιβλιοθήκη καθώς και οι τύποι δεδομένων που χρησιμοποιούνται. 

Πριν περιγράψουμε τις διάφορες συναρτήσεις που περιλαμβάνει η βιβλιοθήκη \selectlanguage{english}CSparse\selectlanguage{greek} και χρησιμοποιούμε κατα την υλοποίηση στα πλαίσια της Διπλωματικής Εργασίας, θα χρειαστεί να ορίσουμε τα είδη των παραμέτρων και των δυνατών τιμών που μπορεί να επιστρέφονται απο την κάθε συνάρτηση.

Κάθε παράμετρος ή τιμή που επιστρέφεται απο τις συναρτήσεις της βιβλιοθήκης, περιγράφεται ως εξής:

\selectlanguage{english}
\begin{itemize}
  \item \textbf{in:} \selectlanguage{greek} Η μνήμη της παραμέτρου δεσμεύεται και η τιμή της ορίζεται πρίν τη κλήση της συνάρτησης και δεν μεταβάλεται κατα την επιστροφή της \selectlanguage{english}
  \item \textbf{in/out:} \selectlanguage{greek} Η μνήμη της παραμέτρου δεσμεύεται και η τιμή της ορίζεται πρίν τη κλήση της συνάρτησης αλλά η τιμή της έχει μεταβληθεί μετά τη κλήση της συνάρτησης \selectlanguage{english}
  \item \textbf{out:} \selectlanguage{greek} Η μνήμη πρέπει να έχει δεσμευτεί πρίν τη κλήση της συνάρτησης. Η τιμή ορίζεται μετά το πέρας της συνάρτησης \selectlanguage{english}
  \item \textbf{work:} \selectlanguage{greek}  Η μνήμη πρέπει να έχει δεσμευτεί πρίν τη κλήση της συνάρτησης. Η τιμή δεν ορίζεται ουτε πρίν ούτε μετά το πέρας της συνάρτησης  \selectlanguage{english}
  \item \textbf{returns:} \selectlanguage{greek} Η επιστρεφόμενη τιμή κάθε συνάρτησης της βιβλιοθήκης. Σε περίπτωση που η συνάρτηση επιστρέφει δείκτη, εαν προκύψει λάθος, επιστρέφεται \selectlanguage{english}NULL\selectlanguage{greek}
\end{itemize}

\selectlanguage{greek} Επίσης να σημειώσουμε οτι σε όλη την υλοποίηση που κάναμε ορίζουμε ως $m$ τον αριθμό των γραμμών και ως $n$ τον αριθμό των στηλών.

Στη συνέχεια θα ορίσουμε τις συναρτήσεις που χρησιμοποιήσαμε απο τη βιβλιοθήκη \selectlanguage{english}CSparse\selectlanguage{greek} μαζί με μια συνοπτική περιγραφή της χρήσης της.

Αρχικά έχουμε την βασική δομή η οποία χρησιμοποιείται για την αναπαράσταση ενός αραιού πίνακα είτε σε μορφή τρίδυμης δομής απεικόνισης \ref{TripletForm} είτε σε Δομή Συμπιεσμένης Στήλης \ref{CCS}

\selectlanguage{english}
\textbf{cs:}
\begin{verbatim}
typedef struct cs_sparse /* matrix in compressed-column or
                            triplet form */
{
    int nzmax ; /* maximum number of entries */
    int m ; /* number of rows */
    int n ; /* number of columns */
    int *p ; /* column pointers (size n+1) or col indices
                (size nzmax) */
    int *i ; /* row indices, size nzmax */
    double *x ; /* numerical values, size nzmax */
    int nz ; /* # of entries in triplet matrix, -1 for
                compressed-col */
} cs ;
\end{verbatim}

\selectlanguage{greek}
Πρόσθεση πινάκων. \\
\selectlanguage{english}
\textbf{cs\_add:} C = \alpha A + \beta B}
\begin{verbatim}
cs *cs_add (const cs *A, const cs *B, double alpha, double beta);
A           in          sparse matrix 
B           in          sparse matrix
alpha       in          scalar
beta        in          scalar
            returns     C=alpha*A+beta*B; NULL on error
\end{verbatim}

\selectlanguage{greek}
Μετατροπή πίνακα $T$ Τρίδυμης Μορφής Απεικόνισης σε πίνακα $C$ με Δομή Συμπιεσμένης Στήλης. Οι στήλες του πίνακα $C$ δεν είναι ταξινομημένες και μπορεί να υπάρχουν διπλές εγγραφές. \\
\selectlanguage{english}
\textbf{cs\_compress:}
\begin{verbatim}
cs *cs_compress (const cs *T) ;
T           in          sparse matrix in triplet form
            returns     C if successful; NULL on error
\end{verbatim}

\selectlanguage{greek}
Αφαίρεση διπλών εγγραφών \\
\selectlanguage{english}
\textbf{cs\_dupl:}
\begin{verbatim}
int cs_dupl (cs *A) ;
A           in/out          sparse matrix; 
            returns     1 if successful; 0 on error
\end{verbatim}

\selectlanguage{greek}
Προσθήκη νέας τιμής σε έναν πίνακα Τρίδυμης Δομής Απεικόνισης. Σε περίπτωση που η μνήμη που έχει δεσμευτεί για τον πίνακα δεν επαρκεί, γίνεται επέκταση της μνήμης \\
\selectlanguage{english}
\textbf{cs\_entry:}
\begin{verbatim}
int cs_entry (cs *T, int i, int j, double x) ;
T           in/out      triplet matrix; new entry added on 
                        output 
i           in          row index of new entry
j           in          column index of new entry
x           in          value of new entry
            returns     1 if successful; 0 on error
\end{verbatim}


\selectlanguage{greek}
Πολλαπλασιασμός αραιών πινάκων.\\
\selectlanguage{english}
\textbf{cs\_multiply: C = AB}
\begin{verbatim}
cs *cs_multiply (const cs* A, const cs *B) ;
A           in          sparse matrix
B           in          sparse matrix
            returns     C = A*B; NULL on error
\end{verbatim}

\selectlanguage{greek}
Εκτύπωση αραιού πίνακα.\\
\selectlanguage{english}
\textbf{cs\_print:}
\begin{verbatim}
int cs_print (const cs* A, int brief) ;
A           in          sparse matrix
brief       in          if 0 print all of A; a few entries 
                        otherwise
            returns     1 if successful; 0 on error
\end{verbatim}

\selectlanguage{greek}
Μετατώπιση αραιού πίνακα.\\
\selectlanguage{english}
\textbf{cs\_transpose:} $C = A^T$
\begin{verbatim}
cs *cs_transpose (const cs* A, int values) ;
A           in          sparse matrix
values      in          if 0 only pattern; both pattern and 
                        values otherwise
            returns     C = A'; NULL otherwise
\end{verbatim}


\selectlanguage{greek}
Δέσμευση μνήμης αραιού πίνακα.\\
\selectlanguage{english}
\textbf{cs\_calloc:}
\begin{verbatim}
void *cs_calloc (int n, size_t size) ;
n         in          number of items to allocate
size      in          size of each item in bytes
          returns     if successful pointer to allocated 
                      block; 
                      NULL on error
\end{verbatim}

\selectlanguage{greek}
Απελευθέρωση μνήμης\\
\selectlanguage{english}
\textbf{cs\_free:}
\begin{verbatim}
void *cs_free (void *p) ;
A           in/out      block to free
            returns     NULL
\end{verbatim}

\selectlanguage{greek}
Δομή για την ανάλυση μεθόδου \selectlanguage{english} Cholesky\selectlanguage{greek} ή \selectlanguage{english} QR \selectlanguage{greek}\\
\selectlanguage{english}
\textbf{css:}
\begin{verbatim}
typedef struct cs_symbolic /* symbolic Cholesky, LU, 
                              or QR analysis */
{
    int *pinv ;         /*  inverse row perm, for QR, 
                            fill perm for Chol */
    int *q ;            /*  fill-reducing column permutation 
                            for LU and QR */
    int *parent ;       /*  elimination tree for Cholesky 
                            and QR */
    int *cp ;           /*  column pointers for Cholesky, 
                            row counts for QR */
    int *leftmost j     /*  leftmost[i] = min(find(A(i,:))), 
                            for QR */
    int m2 ;            /*  # of rows for QR, after adding 
                            fictitious rows */
    double lnz ;        /*  # entries in L for LU or Cholesky; 
                            in V for QR */
    double unz ;        /*  # entries in U for LU; 
                            in R for QR */
} css ;
\end{verbatim}

\selectlanguage{greek}
Δομή απεικόνισης αποτελεσμάτων $LU$ και $QR$ παραγοντοποίησης\\
\selectlanguage{english}
\textbf{csn:}
\begin{verbatim}
typedef struct cs_numeric /* numeric Cholesky, LU, 
                             or QR factorization */
{
    cs *L ;             /*  L for LU and Cholesky, 
                            V for QR */
    cs *U ;             /*  U for LU, R for QR, 
                            not used for Cholesky */
    int *pinv ;         /*  partial pivoting for LU */
    double *B ;         /*  beta [0..n-l] for QR */
} csn ;
\end{verbatim}


\selectlanguage{greek}
Εξάλειψη μηδενικών απο τους αραιούς πίνακες\\
\selectlanguage{english}
\textbf{cs\_dropzeros:}
\begin{verbatim}
int *cs_dropzeros (cs* A) ;
A           in/out      array to remove the zeros from
            returns     if successful 1; 0 otherwise
\end{verbatim}

\selectlanguage{greek}
Παραγοντοποίηση $QR$ αραιού πίνακα $A, n * m$ μεγέθους, όπου $m >= n$\\
\selectlanguage{english}
\textbf{cs\_qr:}
\begin{verbatim}
csn *cs_qr (const cs* A, const css *S) ;
A           in          matrix to factorize
S           in          symbolic analysis from cs_sqr
            returns     numeric factorization N;
                        NULL on error
\end{verbatim}


\selectlanguage{greek}
Αλγεβρική επίλυση $QR$ ή $LU$
\selectlanguage{english}
\textbf{cs\_sqr:}
\begin{verbatim}
csn *cs_sqr (int order, const cs* A, const css *S) ;
order       in          ordering method to use (0 to 3)
A           in          matrix to factorize
qr          in          analyze for QR if nonzero 
                        or LU if zero
            returns     S, symbolic analysis for cs_qr 
                        or cs_lu; NULL on error
\end{verbatim}

