\selectlanguage{greek}
\chapter{Μέθοδοι Υποβιβασμού Τάξης Μοντέλου}
\label{ch:4.chapterMOR}

Όπως τονίσαμε και στην εισαγωγή της διπλωματικής εργασίας (\ref{ch:1.chapterIntroduction}) τα κυκλώματα που εμφανίζονται σε πραγματικές εφαρμογές έχουν φτάσει τα εκατομμύρια στοιχεία. Τα γραμμικά συστήματα που προκύπτουν απο τέτοιας κλίμακας κυκλώματα καθιστούν την συμβατική επίλυση αδύνατη.

Μια απο τις πλέον εδραιωμένες τεχνικές για τη μείωση της πολυπλοκότητας των γραμμικών κυκλωμάτων οι οποίες αποσκοπούν στη μείωση των απαιτούμενων υπολογισμών ή στην απλοποίηση των μαθηματικών μοντέλων είναι οι μέθοδοι υποβιβασμού τάξης μοντέλου. (\textlatin{Model Order Reduction (MOR)}).

Οι μέθοδοι υποβιβασμού τάξης μοντέλου που θα εξετάσουμε στα πλαίσια της παρούσας διπλωματικής εργασίας βασίζονται στους υποχώρους \textlatin{Krylov} ~\ref{KrylovSubspaces}. Τέτοιες μέθοδοι αποτελούν οι \textlatin{Passive Reduced-order Interconnect Macromodeling Algorithm (PRIMA)}~\cite{odabasioglu2003prima} και \textlatin{ Structure-Preserving Reduced-order Interconnect Macromodeling (SPRIM)} ~\cite{freund2004sprim}


\section{Η μέθοδος υποβιβασμού τάξης μοντέλου \textlatin{PRIMA}}

Ο πλεόν βασικός αλγόριθμος για υποβιβασμό τάξης μοντέλου \textlatin{MOR} είναι ο \textlatin{Passive Reduced-order Interconnect Macromodeling Algorithm (PRIMA)}. Μέρος του αλγορίθμου είναι και η μέθοδος \textlatin{Block Arnoldi} που θα εξετάσουμε και υλοποιήσαμε στα πλαίσια της διπλωματικής εργασίας. O \textlatin{Block Arnoldi} αλγόριθμος χρησιμοποιείται για την παραγωγή ορθοκανονικής βάσης για έναν πίνακα μετασχηματισμού.

Πρίν περιγράψουμε τον αλγόριθμο \textlatin{PRIMA} ορίζουμε τους πίνακες οι οποίοι προκύπτουν απο τη μέθοδο \textlatin{MNA}:

\begin{equation} \label{Cxn}
    C x_n = -G x_n + B v_p
\end{equation}
\[
    i_p = L^T x_n
\]
όπου τα διανύσματα $i_p$ και $v_p$ δηλώνουν τις πηγές ρεύματος και τάσης αντίστοιχα
\[
    G = 
    \begin{bmatrix}
    N & E \\
    -E^T & 0
    \end{bmatrix}\\
\]
\[
    C = 
    \begin{bmatrix}
    Q & 0 \\
    0 & H
    \end{bmatrix}\\
\]
\[
    x_n = 
    \begin{bmatrix}
    v \\
    i
    \end{bmatrix}\\
\]

Οι $n * n$ πίνακες $G$ και $C$ αντιπροσωπεύουν τους πίνακες αγωγιμότητας και επιδεκτικότητας. Οι $N$
, $Q$ και $H$ πίνακες περιλαμβάνουν τις αντιστάσεις, πυκνωτές και πηνεία αντίστοιχα. Ο πίνακας $E$ περιλαμβάνει μονάδες, αρνητικές μονάδες και μηδενικά που αντιστοιχούν στις μεταβλητές ρεύματος των νόμων του  \textlatin{Kirchhoff (KCL)}. Στην περίπτωση που οι N-θύρες του κυκλώματος περιλαμβάνουν μόνο θετικά γραμμικά στοιχεία, $Q$, $H$ και $N$ είναι συμμετρικοί, θετικά δηλωμένοι πίνακες (\textlatin{Symmetric Positive Definite (SPD)}). Με βάση το παραπάνω ο $C$ είναι επίσης \textlatin{SPD}. 

Συνεπώς, επιστρέφοντας στην ισότητα \ref{Cxn}, δηλώνουμε:

\begin{equation}
    A = -G^{-1}C
\end{equation}
\begin{equation}
    R = G^{-1}B
\end{equation}


Μετά απο $\left\lfloor\dfrac{q}{N}\right\rfloor +1$ επαναλήψεις (η επιπλέον επανάληψη είναι απαραίτητη μόνο όταν $\frac{q}{N}$ δεν είναι ακέραιος) του \textlatin{PRIMA}, o $n x q$ πίνακας $X$ είναι ο εξής:

\[
    colsp(X) = K_r(A, R,\left\lfloor\dfrac{q}{N}\right\rfloor)
\]
\[
    X^T X = I_q
\]

Στον βασικό αλγόριθμο \textlatin{PRIMA} χρησιμοποιείται μια παραλλαγή της μεθόδου \textlatin{Block Arnoldi} η οποία μας δίνει το εξής:
\[
    (X^T C X)\dot{\widetilde{x_q}} = - (X^T G X)\dot{\widetilde{x_q}} + (X^T B)u_p
\]
\[
    i_p = (L^TX)\dot{\widetilde{x_q}}
\]

To υποβιβασμένης τάξης μοντέλο των \textlatin{MNA} πινάκων είναι:

\[
    \widetilde{C} = X^T C X
\]
\[
    \widetilde{G} = X^T G X
\]
\[
    \widetilde{B} = X^T B
\]
\[
    \widetilde{L} = X^T L
\]

Μετά τον υποβιβασμό του μοντέλου του πίνακα $Y(s)$ στο οποίο αναφερόμαστε ως $\hat{Y}(s)$ είναι πλέον:
\[
    \hat{Y}(s) = \widetilde{L}^T (\widetilde{G} + s \widetilde{C})^{-1} \widetilde{B}
\]

Τυπικά, το μέγεθος των πινάκων $\widetilde{G}$ και $\widetilde{C}$ είναι μικρό με αποτέλεσμα αν είναι εύκολο να βρούμε τους πόλους και τα μηδενικά του $\hat{Y}(s)$.

\section{Η μέθοδος υποβιβασμού τάξης μοντέλου \textlatin{SPRIM}}

Για την περιγραφή του αλγορίθμου \textlatin{SPRIM} θα χρειαστεί να ορίσουμε τους εξής πίνακες:

\[
   G = 
    \begin{bmatrix}
    G_1  & G^T_2\\
    -G_2 & 0
    \end{bmatrix}
\]
\[
    C = 
    \begin{bmatrix}
    C_1  & 0\\
    0 & C_2
    \end{bmatrix}
\]
\[
    B = 
    \begin{bmatrix}
    B_1\\
    0
    \end{bmatrix}
\]

Στο πρώτο βήμα του αλγορίθμου, δημιουργούμε τους εξής πίνακες $A$ και $R$ ως εξής:
\[
    A = (G + sC)^{-1}C
\]
\[
    R = (G + sC)^{-1}B
\]

Στη συνέχεια και ένα απο τα βασικά σημεία του αλγορίθμου είναι η επιλογή της μεθόδου για την παραγωγή του υπο-χώρου \textlatin{Krylov}. Το σημείο αυτό είναι που εισέρχεται και η μελέτη που πραγματοποιήσαμε στα πλαίσια της διπλωματικής εργασίας για την χρήση του Εκτεταμένου \textlatin{Block Arnoldi} αλγορίθμου \ref{EBA}.

Με τη χρήση του επιθυμητού αλγορίθμου λαμβάνουμε το εξής αποτέλεσμα:

\[
    span(V_n) = K_n(A, R)
\]

Συνεχίζουμε με το επόμενο βήμα να είναι ο κατακερματισμός του πίνακα $V_n$ σε κομμάτια (\textlatin{blocks}):

\[
V_n = \begin{bmatrix}
V_1 \\
V_2
\end{bmatrix}
\]

o οποίος μετατρέπεται στην εξής μορφή:

\[
V_n = \begin{bmatrix}
V_1 & 0 \\
0 & V_2
\end{bmatrix}
\]

Υπολογίζουμε τους νέους πίνακες μέσω της ελαχιστοποίησης:

\[
    \widetilde{G_1} = V_q^{T}G_1V_1
\]
\[
    \widetilde{G_2} = V^2_TG_2V_1
\]
\[
    \widetilde{C_1} = V^1_TC_1V_1
\]
\[
    \widetilde{C_2} = V^2_TC_2V_2
\]
\[
    \widetilde{B_1} = V^1_TB_1
\]

και με βάση του πίνακες αυτούς έχουμε:


\[
\widetilde{G_n} = \begin{bmatrix}
\widetilde{G_1} & \widetilde{G_2}^T \\
-\widetilde{G_2}^T & 0
\end{bmatrix}
\]
\[
\widetilde{C_n} = \begin{bmatrix}
\widetilde{C_1} & 0 \\
0 & \widetilde{C_2}
\end{bmatrix}
\]
\[
\widetilde{B_n} = \begin{bmatrix}
\widetilde{B_1} \\
0
\end{bmatrix}
\]

Τέλος, από τους νέους πίνακες υπολογίζεται η συνάρτηση μεταφοράς του συστήματος με
τον εξής τρόπο:
\begin{equation}
    \Tilde{Z_n}(s) = \Tilde{B_n}^T(\Tilde{G_n} + s\Tilde{C_n})^{-1}\Tilde{B_n}
\end{equation}