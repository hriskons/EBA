\begin{titlepage}

\begin{figure}[H]
  \begin{center}
    \includegraphics[width=3cm]{UTH-logo-english.png}
    \label{fig:cover_auth_logo}
  \end{center}
\end{figure}

\centering
\Large Πανεπιστήμιο Θεσσαλίας\\
\Large Πολυτεχνική Σχολή\\
\large Τμήμα Ηλεκτρολόγων Μηχανικών και Μηχανικών Υπολογιστών\\

\vspace{\fill}

\LARGE Προσομοίωση κυκλωμάτων με μεθόδους 
υποβιβασμού τάξης μεγέθους και χρήση εκτεταμένων υποχώρων 
\selectlanguage{english}
Krylov
\selectlanguage{Greek}

\vspace{\fill}

\selectlanguage{english}
\LARGE Circuit Simulation with Model Order Reduction Methods using Extended 
Krylov Subspaces
\selectlanguage{Greek}

\vspace{\fill}

\Large Διπλωματική Εργασία\\
\Large του\\
\Large Χρήστου Κωνσταντά

\vspace{\fill}
\raggedright

\begin{tabular}{ll}
\textbf{Επιβλέπων:} & Γεώργιος Σταμούλης\\
 & Καθηγητής Π.Θ.\\
\end{tabular}

\centering
\vspace{\fill}
\today

\end{titlepage}

\begin{abstract}
Στα πλαίσια της παρούσας διπλωματικής εργασίας θέσαμε σαν στόχο να μελετήσουμε και να πειραματιστούμε μέσω υλοποίησης πάνω σε μεθόδους για τον υποβιβασμό τάξης μοντέλου. Μέθοδοι υποβιβασμού της τάξης των μοντέλων χρησιμοποιούνται εκτεταμένα σε εφαρμογές προσομοίωσης κυκλωμάτων όπου τα γραμμικά συστήματα που απορρέουν από πραγματικά κυκλώματα καθιστούν τη συμβατική επίλυση αδύνατη.

Με την εκτόξευση του μεγέθους των κυκλωμάτων, η συμβατική σχεδίαση και μελέτη της συμπεριφοράς των κυκλωμάτων έγινε μη βιώσιμη. Η λύση για τη σχεδίαση κυκλωμάτων εκατομμυρίων στοιχείων προήλθε μέσω της επιστήμης της μοντελοποίηση κυκλωμάτων τα οποία αποτέλεσαν τη βάση για την εμφάνιση εργαλείων προσομοίωσης κυκλωμάτων.

Η βασική μέθοδος επίλυσης γραμμικών συστημάτων μεγάλης κλίμακας είναι η προσέγγιση της λύσης με όσο το δυνατόν μικρότερο σφάλμα. Ανάλογα με την επιλογή του εκάστοτε αλγορίθμου, εμφανίζεται ένας συμβιβασμός μεταξύ της ταχύτητας και της ακρίβειας επίλυσης.

Οι μέθοδοι που παρουσιάζονται στην παρούσα διπλωματική εργασία βασίζονται πάνω στον υποβιβασμό τάξης μοντέλου \textlatin{(Model Order Reduction (MOR))}. Μια εδραιωμένη μεθοδολογία για τον υποβιβασμό τάξης μοντέλου αποτελεί το ταίριασμα στιγμών \textlatin{(moment-matching)} καθώς είναι αρκετά απλή στην υλοποίηση και παρουσιάζει πολύ καλά αποτελέσματα στην απόδοση.

Οι παραπάνω μέθοδοι ταιριάσματος στιγμών οι οποίες βασίζονται στους τυπικούς υποχώρους \textlatin{Krylov} δεν καταφέρνουν να προσεγγίσουν με καλή ακρίβεια την συμπεριφορά του αρχικού κυκλώματος.

Στα πλαίσια της διπλωματικής εργασίας θα μελετήσουμε τις βασικές μεθόδους ταιριάσματος στιγμών αλλά θα έχουμε και την ευκαιρία να μελετήσουμε μια μέθοδο η οποία εκμεταλλεύεται την ιδιότητα της υπερθεσης \textlatin{(superposition property)} έτσι ώστε να μπορέσει να διαχειριστεί τις πολλαπλές θύρες του κυκλώματος.

Τα αποτελέσματα που παρουσιάζονται μέσω της πρακτικής υλοποίησης μας δίνουν μια πολύ καλή μείωση του λάθους σε σχέση με τις συμβατικές λύσεις που προσεγγίζει το 84\%.
\end{abstract}

\selectlanguage{english}
\begin{abstract}
In this master this we set as a goal to study and experiment through implementation into Model Order Reduction (MOR) techniques. The MOR techniques are intensively being used in circuit simulation where the linear systems that derive from real-world circuit systems are impossible to be solved with conventional methods.

With the advent of circuit systems that contain millions of elements, the conventional methods were deprecated since they could not provide a solution fast enough. The way to approach such linear systems came from the science of circuit modeling that drove the appearance of circuit simulation tools.

The standard approach to solve big order linear systems is to not try and find the exact solution but to approximate it. Depending on the selection of the approximation algorithm we face a tradeoff between accuracy and performance.

The methods that we study in this thesis are based on MOR. A well established methodology to achieve a MOR is the moment-matching technique which offers a simple approach with good performance.

Conventional moment matching techniques that are based on Krylov subspaces come short on achieving adequate approximation on the initial circuit behaviour.

In the thesis we will study the well established momenta matching techniques but we will also implement and experiment on a new approach that takes advantage of the superposition property in order to handle the great number of terminals.

The results that we received from the experiments that we run suggest an error reduction that reaches up to 84\%.

\end{abstract}

\thispagestyle{empty}

\selectlanguage{greek}

\section*{Ευχαριστίες}
\thispagestyle{empty}

Αρχικά θα ήθελα να ευχαριστήσω τους επιβλέποντες καθηγητές της παρούσας διπλωματικής εργασίας κ. Σταμούλη Γεώργιο, κ. Φώτιο Πλέσσα και κ.Αντώνιο Δαδαλιάρη για την ανεκτίμητη υποστήριξη τους στην εκπόνηση της εργασίας.

Ανεκτίμητη ήταν και η βοήθεια του καλού μου φίλου και υποψηφίου διδακτορικού φοιτητή κ.Χρυσόστομου Χατζηγεωργίου ο οποίος με στήριξε και με καθοδήγησε. Ιδιαίτερες ευχαριστίες θα ήθελα να δώσω και στον κ.Γεώργιο Φλώρο ο οποίος έδρασε καταλυτικά.

Τέλος, θα ήθελα να ευχαριστήσω όλη μου την οικογένεια και ιδιαίτερα τη σύντροφο μου Παναγιώτα η οποία έδειξε εξαιρετική υπομονή και στήριξη κατα την πραγματοποίηση της εργασίας.


\clearpage
