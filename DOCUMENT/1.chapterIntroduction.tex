\selectlanguage{greek}
\chapter{Εισαγωγή}
\label{ch:1.chapterIntroduction}

\section{Προσομοίωση κυκλωμάτων στην επιστήμη της πληροφορικής}

Καθώς το μέγεθος των ηλεκτρικών κυκλωμάτων έχει φτάσει σε κλίμακα εκατομμυρίων στοιχείων, η μελέτη συμπεριφοράς τους είναι αδύνατη χωρίς τη χρήση ενός λογισμικού προσομοίωσης του κυκλώματος. Τέτοιου είδος λογισμικά, πραγματοποιούν την αντιστοίχιση του κυκλώματος σε μαθηματικό μοντέλο το οποίο αναπαράγει τη συμπεριφορά του.

Ο στόχος των προσομοιωτών είναι μέσω της τροποποιημένης ανάλυσης κόμβων (\selectlanguage{english}modified nodal analysis\selectlanguage{greek}), να δημιουργήσει εξισώσεις με βάση το ηλεκτρικό κύκλωμα, οι οποίες όταν λυθούν να υπολογίσουν τις τάσεις των κόμβων του κυκλώματος καθώς και τα ρεύματα ορισμένων κλάδων. Οι εξισώσεις ανάγονται σε ένα γραμμικό σύστημα το οποίο χρήζει επίλυσης. 

Για μικρά κυκλώματα η επίλυση τέτοιων συστημάτων είναι εύκολη και η επιλογή μεθόδων επίλυσης γραμμικών συστημάτων ασήμαντη. Όσο τα γραμμικά συστήματα μεγαλώνουν (για ολοκληρωμένα κυκλώματα αλλά και δίκτυα παροχής ηλεκτρικής ενέργειας) η πολυπλοκότητα τους απαιτεί τη χρήση υψηλών υπολογιστικών πόρων. Απαραίτητη γίνεται και η ορθή επιλογή αλγορίθμων επίλυσης καθώς μη αποδοτικοί αλγόριθμοι ή αλγόριθμοι που δεν εκμεταλλεύονται τα ιδιαίτερα στοιχεία του γραμμικού συστήματος, μπορούν να οδηγήσουν είτε σε χρόνους εκτέλεσης απαγορευτικούς, είτε σε μη σταθερά συστήματα επίλυσης.

Για τον λόγο αυτό, η έρευνα στρέφεται προς την αναζήτηση τεχνικών για την βελτίωση των προσομοιωτών, ώστε να επιτυγχάνεται ταχύτερη επίλυση των γραμμικών συστημάτων.

Σημειώνουμε ότι μία από τις συνέπειες που προκύπτουν από διάφορες μεθόδους ταχύτερης επίλυσης γραμμικών συστημάτων είναι η μείωση της ακρίβειας της λύσης. Συνεπώς, προσπαθούμε να βρούμε τη χρυσή τομή μεταξύ ταχύτητας επίλυσης αλλά χωρίς να χαθεί η ακρίβεια.

\section{Η συμβολή της παρούσας διπλωματικής εργασίας}

Στην παρούσα εργασία, χρησιμοποιούμε τεχνικές οι οποίες αντικαθιστούν τα μεγάλης κλίμακας ηλεκτρικά μοντέλα με κάποια πολύς μικρότερης κλίμακας των οποίων η συμπεριφορά είναι παρόμοια με αυτή του αρχικού μοντέλου.

Η παραπάνω διαδικασία ονομάζεται Υποβιβασμός Τάξης Μοντέλου (\textlatin{Model Order Reduction (MOR)}) και θα δοκιμάσουμε την τεχνική \textlatin{Extended Block Arnoldi (ΕΒΑ)} σε σύγκριση με κλασσική μέθοδο Ταιριάσματος Στιγμών (\textlatin{Moment Matching(MM)}) με στόχο να επιτύχουμε μικρότερο λάθος λύση μας.

Μέρος της εργασίας αποτελεί και η υλοποίηση της κλασσικής μεθόδου Ταιριάσματος Στιγμών \textlatin{PRIMA} καθώς και της μεθόδου Εκτεταμένου Υποχώρου \textlatin{Krylov} με τη χρήση του \textlatin{Extended Block Arnoldi} αλγορίθμου και η σύγκριση των λαθών που παράγουν κατά την εύρεση λύσης.

\section{Διάρθρωση της εργασίας}


Αρχικά θα μελετήσουμε το πρόβλημα της Προσομοίωσης Κυκλωμάτων, θα περιγράψουμε την μετατροπή των κυκλωματικών στοιχείων σε γραμμικά συστήματα μέσω της Τροποποιημένης Ανάλυσης Κόμβων (\textlatin{Modified Nodal Analysis (MNA)}) ενώ θα αναφερθούμε και στην Ανάλυση Συνεχούς.

Στη συνέχεια είναι απαραίτητο να ορίσουμε το Μαθηματικό Υπόβαθρο μέρους του οποίου είναι οι Αραιοί Πίνακες, οι ιδιότητες τους καθώς και η απεικόνιση τους στα υπολογιστικά συστήματα. Θα συνεχίσουμε με μια σύντομη αναφορά στην Αλγοριθμική επίλυση των αραιών πινάκων καθώς και διάφορες μεθόδους παραγοντοποίησης τους. Τέλος, θα γίνει αναφορά στους Υποχώρους \textlatin{Krylov} οι οποίοι αποτελούν βάση του αλγορίθμου που θα εξετάσουμε στα πλαίσια της διπλωματικής εργασίας.

Στο επόμενο κεφάλαιο, θα εξετάσουμε τις βασικές μεθόδους για τον υποβιβασμό τάξης μοντέλου (\textlatin{Model Order Reduction (MOR)}). Πιο συγκεκριμένα, θα αναφερθούμε στη μέθοδο \textlatin{PRIMA} η οποία συνθέτει τον πλέον βασικό αλγόριθμο για τον υποβιβασμό τάξης ενός μοντέλου ενώ θα αναλύσουμε και την μέθοδο \textlatin{SPRIM} εξίσου θεμελιώδους στην θεωρία υποβιβασμού τάξης μοντέλου.

Ένα από τα βασικά κεφάλαια στα πλαίσια της παρούσας εργασίας είναι το Κεφάλαιο 5 στο οποίο αναλύουμε τον αλγόριθμο \textlatin{Block Arnoldi (BA)} ο οποίος χρησιμοποιείται για την κατασκευή του υποχώρου \textlatin{Krylov}. Στη συνέχεια, θα αναλύσουμε την επέκταση του παραπάνω αλγορίθμου τον \textlatin{Extended Block Arnoldi} που παράγει τον εκτεταμένο υποχώρο \textlatin{Krylov} ο οποίος θα χρησιμοποιηθεί για να ενισχύσουμε την ακρίβεια στη μέθοδο Ταιριάσματος Στιγμών (\textlatin{Moment Matching (MM)} και να πετύχουμε καλύτερα αποτελέσματα.

Με την ολοκλήρωση των παραπάνω κεφαλαίων έχουμε ολοκληρώσει τη θεωρητική βάση της παρούσας εργασίας. Στο Κεφάλαιο 6 περνάμε στην Πειραματική Διαδικασία στην οποία αναφερόμαστε στο πρακτικό κομμάτι της υλοποίησης που πραγματοποιήθηκε στα πλαίσια της εργασίας. Αρχικά αναφερόμαστε στα αρχεία εισόδου και την μορφή που έχουν, συνεχίζουμε με την αναφορά στην βιβλιοθήκη \textlatin{CSparse} και αναφέρουμε κάποιες από τις επιπλέον συναρτήσεις που χρειάστηκε να κατασκευάσουμε. Σημαντικό κομμάτι της εργασίας στάθηκε και η ανάκτηση του παραγοντοποιημένου πίνακα \textlatin{Q} απο τα διανύσματα \textlatin{Householder} που παράγει η μέθοδος \textlatin{QR} της βιβλιοθήκης \textlatin{CSParse}. Τέλος, στο κεφάλαιο αυτό, γίνεται εκτενής αναφορά στα αποτελέσματα που λάβαμε κατά την πειραματική διαδικασία.
